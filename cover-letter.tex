\documentclass[12pt]{letter}
\usepackage{graphicx}
% \setlength{\parindent}{0pt}% Remove paragraph indent
\usepackage{url}

\usepackage[margin=1.0in,top=0.5in,bottom=0.5in]{geometry}

\begin{document}
\thispagestyle{empty}

\emph{To whom it may concern,}
\par\medskip

My name is Ivan, I am 36 years old and I have been doing data science since 2009. I have collaborated or led applied projects and developed solutions for customers, such as Huawei, Bosch, Airbus, and Sber, that were related to challenging industrial problems: digital signal predistortion, predictive analytics, time series forecasting (ICMLA 2019, DSAA 2019), diagnostics of respiratory illnesses from audio recordings (IEEE JSTSP 2022), and reinforcement learning in operations research and resource allocation.%
\footnote{
    \url{https://www.linkedin.com/in/ivan-nazarov-236b08138/}
}

Since 2014 I have been steadily leaning towards scientific research in machine learning and applied mathematics. Throughout my experience I have made research contributions, including GANs for steganography (ICMV 2019), variational sparsification for Complex-valued deep networks (ICML 2020), decentralized One-class SVM for anomaly detection, sparse inductive matrix completion, and etc.%
\footnote{
    % cannot update my email due to trade controls, sadge :(
    \url{https://scholar.google.ru/citations?user=lF5HI3QAAAAJ}
    % 
    % \url{https://proceedings.mlr.press/v119/nazarov20a.html}
}

I graduated from the Master's programme on computer science at the National Research University Higher School of Economics in 2016. In 2020 I completed my PhD studies in applied mathematics and computer science at Skolkovo Institute of Science and Technology, where my thesis research focused on sparse-regularized matrix decompositions, variational dropout for Complex-valued networks, and parameter pruning based on second-order approximations. Since graduation I have continued to improve my skills in software engineering and deep learning while deepening knowledge of computer science, optimization, and probability.

I am a strong proponent of adopting modern devops practices to scientific research. To this end my advanced python skills, coupled with hands-on experience with C/Cython extensions, torch, and Jax, allow me to quickly iterate and generate reproducible scientific and business value.%
\footnote{
    \url{https://github.com/ivannz}
}
Indeed, version-controlled experiments and artifacts, methodical ablations and aptly visualized results are crucial for evaluating hypotheses and placing the findings within a bigger picture. I firmly believe that a scientific paper should reflect the research process as it is: a journey from the motivation to a verifiable discovery, that has internal logic in its systematic inching through healthy self-criticism and meticulous comparisons against the related work.
% 
% I have no problem with engaging with complex concepts in probability and statistics, optimization and numerical methods, differential geometry and linear algebra. I feel secure in my ability to distill and combine key ideas from diverse research fields, and enjoy communicating findings through papers, presentations and teaching materials.

In 2021 my team won the NeurIPS Nethack Challenge for our hybrid neural-algorithmic hierarchical agent, and this encouraged me to seek more impactful uses of reinforcement learning. Since then, I have been captivated by the synergy between machine learning approaches, particularly reinforcement learning, and classic search algorithms for combinatorial optimization and adaptive computations. In my opinion, efficient fine-tuning of solution techniques on practically relevant problems promises tangible advances in algorithm design, resource allocation, responsive social welfare and medicine.

% I intend to continue pursing research in Machine Learning and RL and I am confident that my software engineering skills, understanding of modern DEVops practices, and diligent and broad approach to research, coupled with excitement with AI in general, would allow me to make meaningful contributions both the field itself and its applications.

% Currently, I am particularly interested in reinforcement learning (RL) and optimal control. I am researching two directions: the problem of sub-policy discovery in hierarchical reinforcement learning, and intrinsic motivation mechanisms for meaningful exploration in procedurally generated environments. I am also pondering the idea of fusing adaptive computation approaches with the policy improvement via Monte Carlo Tree Search (MuZero + PonderNet). At the same time the following topics are also interesting to me due to my past experience: deep network sparsification and pruning, optimisation on matrix manifolds, Variational Bayes, Bayesian neural networks and Gaussian processes, active learning, statistical learning, kernel methods.

% Currently, I my research is focused on applications of machine learning to combinatorial optimization, however, due to my prior experience, I am also interested in following topics: hierarchical policies and efficient representations in reinforcement learning, model sparsification and pruning, Variational Bayes, and optimisation on matrix manifolds.

% \emph{Thank you for the opportunity.}
\emph{Thank you for the opportunity, and I am looking forward to speaking with you soon.}
\par
\hfill \emph{Sincerely, Ivan}
\end{document}
