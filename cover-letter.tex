\documentclass[14pt]{letter}
\usepackage{graphicx}% http://ctan.org/pkg/graphicx
\setlength{\parindent}{0pt}% Remove paragraph indent
\usepackage{url}

\begin{document}
\thispagestyle{empty}

%\hspace*{0.55\linewidth}
%\begin{minipage}{0.45\linewidth}
%Dmitry I. Ignatov \par
%Faculty of Computer Science \par
%National Research University Higher School of Economics, Moscow
%\end{minipage}\par \bigskip
%
% \includegraphics[scale=0.55]{logo_hse_cmyk_e.eps}

\vspace{1.5cm}

\begin{minipage}{0.5\linewidth}
To whom it may concern, \par
\end{minipage} \par\bigskip

My name is Ivan Nazarov. I am 35 years old and have with 12 year working experience. I have
been deeply immersed in Machine Learning for the last seven years, during which worked on
various projects: applied, tackling industrial problems such as forecasting and signal processing,
and theoretically inclined, including optimization, matrix decompositions, variational Bayes
and sparsification methods for DL. \par\medskip

In 2016 I graduated from the MSc programme on computer science at the National Research
University Higher SChool of Economics and in 2020 I successfully completed my PhD studies
on applied mathematics, computer science and engineering at Skoltech. The thesis is on
the topic of model sparsification: sparse-regularized matrix decompositions, variational
dropout, and parameter pruning methods based on second-order loss approximation. I have
a paper about variational Dropout for complex-valued deep networks at ICML2020%
\footnote{
    \url{https://proceedings.mlr.press/v119/nazarov20a.html}
} and co-authored several other publications in other venues.%
\footnote{
    \url{https://scholar.google.ru/citations?user=lF5HI3QAAAAJ}
}
\par\medskip

I write a lot of code in python%
\footnote{
    \url{https://github.com/ivannz}
}
and, when the problem demands, in C/C++. I know how to plan and set up numerical experiments,
have no problem with engaging with probability and mathematical statistics, optimization and
numerical methods, functional analysis, and linear algebra. When doing research I have a broader,
integral look at the methods, their interactions and relations to adjacent ML and RL subfields.
I feel secure in distilling the key ideas from diverse research fields and combining them,
and enjoy communicating them to others through papers, presentations and teaching materials.
\par\medskip

In the last year I have been consumed by the exciting and challenging field of Reinforcement
Learning and Optimal Control, specifically hierarchical policies, planning, MCTS and adaptive
computations. My team took first place in the recent NeurIPS Nethack Challenge for our hybrid
algorithmic-neural solution (Team RAPH). I intend to continue pursing research in RL and 
that my software engineering skills and diligent and broad approach to research, coupled
with excitement with AI in general, would allow me to make meaningful contributions both
the field itself and its applications. \par\medskip

% My current interests lie with hierarchical reinforcement learning, optimal control,
% planning and intrinsic motivation. Recently, I have been keen on fusing ideas from
% adaptive computation with option discovery and planning. At the same time, I am
% pondering the idea of viewing curiosity-based intrinsic motivation as self-supervised
% rewarding mechanism.


\hfill Thank you for you consideration,\par%
\hfill Ivan Nazarov

\end{document}
