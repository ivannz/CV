\documentclass[14pt]{letter}
\usepackage{graphicx}% http://ctan.org/pkg/graphicx
\setlength{\parindent}{0pt}% Remove paragraph indent
\usepackage{url}

\begin{document}
\thispagestyle{empty}

%\hspace*{0.55\linewidth}
%\begin{minipage}{0.45\linewidth}
%Dmitry I. Ignatov \par
%Faculty of Computer Science \par
%National Research University Higher School of Economics, Moscow
%\end{minipage}\par \bigskip
%
% \includegraphics[scale=0.55]{logo_hse_cmyk_e.eps}

\vspace{1.5cm}

\begin{minipage}{0.5\linewidth}
To whom it may concern, \par
\end{minipage} \par\bigskip

My name is Ivan Nazarov. I am 35 years old and have with 12 year working experience. I have
been deeply immersed in Machine Learning for the last seven years, during which worked on
various projects: applied, tackling industrial problems such as forecasting and signal processing,
and theoretically inclined, including optimization, matrix decompositions, variational Bayes
and sparsification methods for DL. \par\medskip

In 2016 I graduated from the MSc programme on computer science at the National Research
University Higher School of Economics and in 2020 I successfully completed my PhD studies
on applied mathematics, computer science and engineering at Skoltech. The thesis is on
the topic of model sparsification: sparse-regularized matrix decompositions, variational
dropout, and parameter pruning methods based on second-order loss approximation. I have
a paper about variational Dropout for complex-valued deep networks at ICML2020%
\footnote{
    \url{https://proceedings.mlr.press/v119/nazarov20a.html}
} and co-authored several other publications in other venues.%
\footnote{
    \url{https://scholar.google.ru/citations?user=lF5HI3QAAAAJ}
}
\par\medskip

I write a lot of code in python%
\footnote{
    \url{https://github.com/ivannz}
}
and, when the problem demands, in C/C++. I know how to set up numerical experiments, plan
and carry out albation studies, and communicate the findings spcialized and broader audiences.
I have no problem with engaging with probability and mathematical statistics, optimization and
numerical methods, functional analysis, and linear algebra. When doing research I have a broader,
integral look at the methods, their interactions and relations to adjacent ML and RL subfields.
I feel secure in distilling the key ideas from diverse research fields and combining them,
and enjoy communicating them to others through papers, presentations and teaching materials.
\par\medskip

In the last two years I have been consumed by the exciting and challenging field of Reinforcement
Learning and Optimal Control, specifically hierarchical policies, planning, MCTS and adaptive
computations. My team took first place in the recent NeurIPS Nethack Challenge for our hybrid
algorithmic-neural solution (Team RAPH).

Recently I was intrigued by the prospect of marrying classic search algorithms with the latest
advances in machine and deep learning and the problem of generalizing from small problems
to large ones. In particular, improving the runtime speed by practical methods to learn heuristics
for Branch and Bound solvers from a collection of small Mixed Integer Program instances with
further fine-tuning on a handful of practically relevant problems within the same class.

I intend to continue pursing research in RL and  that my software engineering skills and
diligent and broad approach to research, coupled with excitement with AI in general, would
allow me to make meaningful contributions both the field itself and its applications.
\par\medskip

% Currently, I am particularly interested in reinforcement learning (RL) and optimal control. I am researching two directions: the problem of sub-policy discovery in hierarchical reinforcement learning, and intrinsic motivation mechanisms for meaningful exploration in procedurally generated environments. I am also pondering the idea of fusing adaptive computation approaches with the policy improvement via Monte Carlo Tree Search (MuZero + PonderNet). At the same time the following topics are also interesting to me due to my past experience: deep network sparsification and pruning, optimisation on matrix manifolds, Variational Bayes, Bayesian neural networks and Gaussian processes, active learning, statistical learning, kernel methods.

\hfill Thank you for you consideration,\par%
\hfill Ivan Nazarov

\end{document}
